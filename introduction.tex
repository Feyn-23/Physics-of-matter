\chapter{Fondamenti di Fisica Statistica}
\section{Introduzione}

Nello studio di sistemi fisici può capitare di trovarsi di fronte ad un ammasso di particelle, classiche o quantistiche, che si muovono in un volume, aventi una determinata energia. Sorge quindi il quesito di come studiare un sistema di questo tipo; dato ad esempio in un sistema composto da un piccolo numero di particelle, risulta abbastanza naturale utilizzare un approccio meccanico, nel quale viene studiato separatamente il moto delle signole particelle. Quando però il numreo di elementi costituenti il sistema cresce ci si rende conto che la difficoltà a studiare un sistema di questo tipo aumenta. Ad esempio, nel caso dello studio di una mole di gas (ideale o relae che sia) comporta lo studio della dinamica di $6.022\times 10^{23}$ molecole. Risulta quindi necessario l'utilizzo di un diverso metodo al fine di semplificare lo studio di un tale sistema, senza però perdere l'accuratezza nello studio delle sue proprietà.
\\
A tal fine, consideriamo un approccio termodinamico a tale problema, in quanto la termodinamica è quella parte della fisica che permetto lo studio del mondo \textit{macroscopico} sulla base di postulati intuitivi e leggi fenomenologiche (sulla base di postulati intuitivi e leggi fenomenologiche (la teoria cinetica dei gas è il solo caso in cui la termodinamica può essere derviata, quasi interamente, da "principi primi"). Una equazione come la legge dei gas perfetti permette, ad esempio, lo studio di un sistema senza dover ricorrere direttamente allo studio \textit{microscopico} del sistema in questione. Il modello termodinamico non corrisponde in maniera rigorosa al mondo fisico reale, perché ne ignora la struttura atomica della materia.
\\
Un terzo approccio possibile, che i  realtà è molto legato a quello termodinamico, si vedrà infatti un legame profondo tra i due, è quello della meccanica statistica. Questa, permette di calcolare le proprietà macroscopiche di un sistema dalla distribuzione statistica del comportamento microscopico individuale di atomi e molecole, studiandone così il comportamento medio. Quello che questa teoria ci permette di fare e è, invece di seguire le traiettorie di ogni singola molecole e risolvere le equazioni del moto (classiche o quantistiche), ci si affida ad una descrizione statistica, ovvero ottenere informazioni macroscopiche (e.g. T o P) del sistema analizzando comportamenti microscopici collettivi.
\\
Al fine di una migliore compresnione di quello che veine dopo, segue una rapida sezione che ripdrende i concetti fondamentali della probabilità.
\section{Probabilità}
